\documentclass[12pt]{article}
\usepackage{graphicx}
\usepackage{times} 
\usepackage{amsmath,multicol}
\usepackage[a4paper,top=1cm,left=1.2in,bottom=5cm, right=1.4in,]{geometry}
\usepackage{enumitem}
\usepackage{fancybox}
\usepackage{fancyhdr}
\usepackage[dvipsnames]{xcolor}
\usepackage{url}
\pagestyle{fancy}
\fancyhf{}
\renewcommand{\headrulewidth}{2.3pt} 
\renewcommand{\headrule}{{\color{MidnightBlue}\hrule width\headwidth height\headrulewidth}}
\fancyhead[L]{\includegraphics[width=5cm,height=2cm]{COMET.jpeg}} 
\fancyhead[C]{\textcolor{MidnightBlue}{\Large\textbf{GATE  \vspace{0.3em} \\ \large EC 2009-38}}}
\fancyhead[R]{\textcolor{MidnightBlue}{K SATHVIKA \\ COMET.FWC20}} 
\setlength{\headheight}{6cm}
\setlength{\headsep}{19pt}
\begin{document}
\begin{multicols}{2}

\noindent\textbf{Abstract} \\[0.5em]
\textit{Simulation of latch behavior using Raspberry Pi Pico to demonstrate NAND and NOR latch transitions for the input combinations (0,1) → (1,1).}

\vspace{1em}
\noindent\textbf{1. Components}
\begin{table}[h]
\small
\centering
\begin{tabular}{|p{4.2cm}|c|}
\hline
\textbf{Component} & \textbf{Qty} \\
\hline
 Pico2w & 1 \\
Push Buttons & 2 \\
LEDs & 2 \\
220$\Omega$ Resistors & 4 \\
Breadboard & 1 \\
Jumper Wires & 10 \\
Laptop with Thonny IDE & 1 \\
\hline
\end{tabular}
\caption*{Table: Components used}
\end{table}

\vspace{1em}
\noindent\textbf{2. Setup}
\begin{itemize}
    \item GP15: Input P1 (Push Button)
    \item GP14: Input P2 (Push Button)
    \item GP16: NAND Q Output (LED)
    \item GP17: NOR Q Output (LED)
    \item GND and VBUS properly connected
\end{itemize}

\vspace{1em}
\noindent\textbf{3. Observation}
\begin{itemize}
    \item \textbf{NAND Latch:} (0,1) → (1,0) → holds at (1,0)
    \item \textbf{NOR Latch:} (0,1) → (1,0) → transitions to (0,0)
\end{itemize}

\end{multicols}
\\
\noindent\textbf{4. Truth Tables}\\
\\
\\
\vspace{2em}
\begin{minipage}{0.45\linewidth}
\centering
\textbf{NAND Latch} \\
\[
\begin{array}{|c|c|c|}
\hline
P1 & P2 & \text{Output (Q1, Q2)} \\
\hline
0 & 1 & (1, 0) \\
1 & 1 & (1, 0) \text{ (holds)} \\
\hline
\end{array}
\]
\end{minipage}
\hfill
\begin{minipage}{0.45\linewidth}
\centering
\textbf{NOR Latch} \\
\[
\begin{array}{|c|c|c|}
\hline
P1 & P2 & \text{Output (Q1, Q2)} \\
\hline
0 & 1 & (1, 0) \\
1 & 1 & (0, 0) \\
\hline
\end{array}
\]
\end{minipage}

\vspace{20em}
\newpage
\noindent\textbf{5. Circuit Image} \\
\\
\includegraphics[width=0.6\linewidth]{assembly.jpeg}


\vspace{1em}
\noindent\textbf{6. GitHub Code Link} \\
\url{https://github.com/sathvi2710/fwc/blob/main/hardware/platformio/plaformio.py}

\vspace{1em}
\noindent\textbf{7. Conclusion}

This project successfully demonstrates latch behavior for NAND and NOR gates using MicroPython and Raspberry Pi Pico. 

\end{document}