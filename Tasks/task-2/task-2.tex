\documentclass[12pt]{article}
\usepackage{graphicx}
\usepackage{times} 
\usepackage{amsmath,multicol}
\usepackage[a4paper,top=1cm,left=1.2in,bottom=5cm, right=1.4in,]{geometry}
\usepackage{enumitem}
\usepackage{fancybox}
\usepackage{fancyhdr}
\usepackage[dvipsnames]{xcolor}
\pagestyle{fancy}
\fancyhf{}
\renewcommand{\headrulewidth}{2.3pt} 
\renewcommand{\headrule}{{\color{cyan}\hrule width\headwidth height\headrulewidth}}
\fancyhead[L]{\includegraphics[width=5cm,height=2cm]{COMET.jpeg}} 
\fancyhead[C]{\textcolor{MidnightBlue}{\Large\textbf{NCERT \vspace{0.3em} \\ \large Task-3}}}
\fancyhead[R]{\textcolor{MidnightBlue}{K SATHVIKA \\ COMET.FWC20}} 
\setlength{\headheight}{4cm}
\setlength{\headsep}{15pt}
\begin{document}

\noindent\textbf{Q.12} \textbf{For the output F to be 1 in the logic circuit shown, the  the input combination should be:}

\vspace{1em}
\begin{center}
    \includegraphics[width=0.6\textwidth]{circuit.png}
\end{center}
\vspace{1em}

\noindent\textbf{Options:}
\begin{multicols}{2}
\begin{enumerate}[label=(\Alph*)]
    \item \( A = 1,\ B = 1,\ C = 0 \)
    \item \( A = 1,\ B = 0,\ C = 0 \)
    \item \( A = 0,\ B = 1,\ C = 0 \)
    \item \( A = 0,\ B = 0,\ C = 1 \)
\end{enumerate}
\end{multicols}
\vspace{1em}
\noindent\textbf{Solution:}

\begin{itemize}
    \item The circuit contains two logic gates receiving inputs A, B, and C. From the diagram:
    \begin{itemize}
        \item The first gate is an OR gate taking inputs \( A \) and \( B \): output = \( A + B \)
        \item The second gate is a NOR gate taking the same inputs \( A \) and \( B \): output = \( \overline{A + B} \)
        \item These two outputs are fed into an XOR gate: output = \( (A + B) \oplus \overline{A + B} \)
        \item The result of the XOR is then passed into an OR gate with input \( C \): output \( F = [(A + B) \oplus \overline{A + B}] + C \)
    \end{itemize}

    \item Now simplify:
    \[
    (A + B) \oplus \overline{A + B} = 1 \quad \text{(since any value XOR its complement is 1)}
    \]
    \[
    F = 1 + C = 1 \quad \text{(since OR with 1 gives 1)}
    \]

    \item So, for any values of A and B (as long as the circuit logic is valid), the output of the XOR will be 1, and OR-ing it with any \( C \) gives \( F = 1 \). Hence, **all options result in F = 1**.

    \item However, the question asks: \emph{“the input combination should be”} — implying one valid combination is sufficient.
    
    \item Option (B): \( A = 1,\ B = 0,\ C = 0 \)
    \[
    A + B = 1,\quad \overline{A + B} = 0,\quad \text{XOR} = 1,\quad F = 1 + 0 = 1
    \]
    \item So, this is correct.

\end{itemize}

\vspace{1em}
\noindent\doublebox{
    \textbf{Correct answer: (B) \( A = 1,\ B = 0,\ C = 0 \)}
}

\end{document}