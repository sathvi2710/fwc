\documentclass[conference]{IEEEtran}
\usepackage{cite}
\usepackage{amsmath,amssymb}
\usepackage{graphicx}
\usepackage{booktabs}
\usepackage{hyperref}
\usepackage{float}

\hypersetup{
    colorlinks=true,
    linkcolor=black,
    urlcolor=blue,
    citecolor=black
}

\title{5G NR Layer-2 PDU Structure Definitions\\
\large A 3GPP-Compliant C Implementation}

\author{
\IEEEauthorblockN{Sathvika Koppu}
\IEEEauthorblockA{COMETFWC020\\
Project Intern,IIITB COMET\\
India\\
Email: sathvikareddy2710@gmail.com
}
}

\begin{document}
\maketitle

% ================= ABSTRACT =================
\begin{abstract}
Fifth Generation New Radio (5G NR) introduces a flexible and service-oriented protocol architecture to support enhanced Mobile Broadband (eMBB), Ultra-Reliable Low-Latency Communications (URLLC), and massive Machine-Type Communications (mMTC).
Layer-2 protocols play a critical role in ensuring reliability, quality of service, and efficient radio resource utilization.

This paper presents a comprehensive and bit-accurate implementation of 5G NR Layer-2 Protocol Data Unit (PDU) header structures using the C programming language, strictly adhering to 3GPP Release 15 and Release 16 specifications.
The implementation covers SDAP, PDCP, RLC, and MAC layers and is intended for academic learning, protocol understanding, and research prototyping.
\end{abstract}

\begin{IEEEkeywords}
5G NR, Layer-2, SDAP, PDCP, RLC, MAC, PDU, 3GPP, C Programming
\end{IEEEkeywords}

% ================= INTRODUCTION =================
\section{Introduction}
5G New Radio (NR) represents a major evolution in cellular communication systems, designed to meet diverse performance requirements such as high throughput, ultra-low latency, and massive device connectivity.
These capabilities are enabled by a modular protocol stack defined by the 3rd Generation Partnership Project (3GPP).

Layer-2 of the 5G NR protocol stack is responsible for data adaptation, security, reliability, segmentation, reordering, and multiplexing.
Although the 3GPP specifications describe these protocols in detail, their PDU formats are complex and difficult to interpret directly.
This project bridges the gap between specification and implementation by translating Layer-2 PDU definitions into precise C structures.

% ================= MOTIVATION =================
\section{Motivation}
Understanding real-world protocol behavior requires more than theoretical knowledge.
Students and early-stage engineers often struggle to visualize how PDUs are structured at the bit level.
This project was motivated by the need for:
\begin{itemize}
    \item A clear mapping between 3GPP diagrams and actual code
    \item A reusable reference for protocol stack learning
    \item Interview and placement-oriented preparation for telecom roles
\end{itemize}

% ================= LAYER-2 OVERVIEW =================
\section{5G NR Layer-2 Protocol Stack}

\begin{figure}[H]
\centering
\includegraphics[width=0.9\columnwidth,height=10cm]{ps.png}
\caption{5G NR Layer-2 Protocol Stack}
\label{fig:layer2stack}
\end{figure}

The 5G NR Layer-2 stack consists of four primary protocols:
\begin{itemize}
    \item SDAP (Service Data Adaptation Protocol)
    \item PDCP (Packet Data Convergence Protocol)
    \item RLC (Radio Link Control)
    \item MAC (Medium Access Control)
\end{itemize}

Each protocol performs a distinct function and adds its own PDU header before passing data to the lower layer.

% ================= PDU FLOW =================
\section{Layer-2 PDU Processing Flow}

\begin{figure}[H]
\centering
\includegraphics[width=0.9\columnwidth]{pdu_flow.png}
\caption{Layer-2 PDU Formation and Flow}
\label{fig:pduflow}
\end{figure}

User data enters the Layer-2 stack at the SDAP layer and is progressively encapsulated with protocol-specific headers before transmission over the physical layer.

% ================= SDAP =================
\section{SDAP PDU Design}
SDAP maps QoS flows to Data Radio Bearers (DRBs).
The implemented SDAP PDUs include:
\begin{itemize}
    \item SDAP Data PDU with Reflective QoS Indicator (RQI) and Reflective DRB Indicator (RDI)
    \item SDAP Data PDU without RQI and RDI
\end{itemize}

These PDUs are implemented using compact 1-byte headers, ensuring minimal overhead.

% ================= PDCP =================
\section{PDCP PDU Design}
PDCP provides security, header compression, and reordering.
This project supports:
\begin{itemize}
    \item PDCP Data PDUs with 12-bit and 18-bit sequence numbers
    \item PDCP Control PDUs for status reporting
    \item PDCP Control PDUs for ROHC feedback
\end{itemize}

Bit-fields are used to accurately represent PDCP header formats as shown in the specification.

% ================= RLC =================
\section{RLC PDU Design}
RLC ensures segmentation, reassembly, and reliability.
The following modes are implemented:
\begin{itemize}
    \item Unacknowledged Mode (UM) – 6-bit and 12-bit SN
    \item Acknowledged Mode (AM) – 12-bit and 18-bit SN
    \item RLC AM STATUS PDUs
\end{itemize}

Segmented and complete PDUs are modeled separately for clarity.

% ================= MAC =================
\section{MAC PDU Design}
The MAC layer handles multiplexing, scheduling, and priority handling.
Implemented MAC subheaders include:
\begin{itemize}
    \item Short and long MAC subheaders
    \item Fixed-size MAC Control Element subheaders
    \item MAC subheaders with extension fields
\end{itemize}

These definitions closely follow TS 38.321.

% ================= IMPLEMENTATION =================
\section{Implementation Methodology}
All PDUs are implemented using C bit-fields.
Multi-byte fields are reconstructed using:
\begin{equation}
SN = (SN_{high} \ll 8) | SN_{low}
\end{equation}

Compiler dependency of bit-fields is documented as a limitation.

% ================= OUTPUT =================
\section{Testing and Output Verification}

\begin{figure}[H]
\centering
\includegraphics[width=0.9\columnwidth]{output.png}
\caption{Program Output Showing PDU Structure Sizes}
\label{fig:output}
\end{figure}

A test program verifies the implementation by printing structure sizes using the \texttt{sizeof()} operator.

% ================= APPLICATIONS =================
\section{Applications}
\begin{itemize}
    \item Academic learning and protocol visualization
    \item 5G protocol simulation
    \item SDR-based experimentation
    \item Telecom placement preparation
\end{itemize}

% ================= LIMITATIONS =================
\section{Limitations}
\begin{itemize}
    \item Compiler-dependent bit-field layout
    \item No full packet encoding or decoding
    \item Not suitable for production systems
\end{itemize}

% ================= FUTURE WORK =================
\section{Future Enhancements}
\begin{itemize}
    \item Full encoder and decoder implementation
    \item Integration with GNU Radio and SDRs
    \item Support for future 3GPP releases
\end{itemize}

% ================= CONCLUSION =================
\section{Conclusion}
This paper presents a detailed and specification-compliant implementation of 5G NR Layer-2 PDU header structures in C.
The project effectively bridges theoretical 3GPP documentation and practical implementation, making it valuable for students, researchers, and engineers.

% ================= REFERENCES =================
\begin{thebibliography}{00}

\bibitem{sdap}
3GPP TS 37.324, ``Service Data Adaptation Protocol (SDAP) specification.''

\bibitem{pdcp}
3GPP TS 38.323, ``Packet Data Convergence Protocol (PDCP) specification.''

\bibitem{rlc}
3GPP TS 38.322, ``Radio Link Control (RLC) protocol specification.''

\bibitem{mac}
3GPP TS 38.321, ``Medium Access Control (MAC) specification.''

\end{thebibliography}

\end{document}
