\documentclass[11pt,a4paper]{article}

% -------------------- Packages --------------------
\usepackage[margin=1in]{geometry}
\usepackage{hyperref}
\hypersetup{hidelinks}
\usepackage{graphicx}
\usepackage{listings}
\usepackage{xcolor}
\usepackage{booktabs}
\usepackage{longtable}
\usepackage{titlesec}

% -------------------- Code Style --------------------
\lstset{
    language=C,
    basicstyle=\ttfamily\small,
    keywordstyle=\color{blue},
    commentstyle=\color{gray},
    stringstyle=\color{red},
    numbers=left,
    numberstyle=\tiny,
    stepnumber=1,
    breaklines=true,
    frame=single
}

% -------------------- Title --------------------
\title{\textbf{5G NR Layer-2 PDU Structure Definitions}\\
\vspace{0.3cm}
\large A 3GPP-Compliant C Implementation}
\author{\textbf{Sathvika Koppu}}
\date{2025}

% -------------------- Document --------------------
\begin{document}
\maketitle

\tableofcontents
\newpage

% -------------------- Overview --------------------
\section{Overview}
This project provides a comprehensive implementation of \textbf{5G New Radio (NR) Layer-2 Protocol Data Unit (PDU) header structures} in the C programming language.  
All definitions are strictly derived from official \textbf{3GPP Release 15/16 specifications} and are designed to be bit-accurate.

The implementation covers the following Layer-2 protocols:
\begin{itemize}
    \item Service Data Adaptation Protocol (SDAP)
    \item Packet Data Convergence Protocol (PDCP)
    \item Radio Link Control (RLC)
    \item Medium Access Control (MAC)
\end{itemize}

% -------------------- Features --------------------
\section{Features}
\begin{itemize}
    \item 24+ PDU header structures across all Layer-2 protocols
    \item Bit-field accurate representations matching 3GPP specifications
    \item Multiple variants for different sequence number lengths
    \item Extensive in-code documentation
    \item Zero external dependencies (pure C implementation)
\end{itemize}

% -------------------- Structure Coverage --------------------
\section{Structure Coverage}

\subsection{SDAP Layer (TS 37.324)}
\begin{itemize}
    \item SDAP Data PDU with RQI and RDI
    \item SDAP Data PDU without RQI and RDI
\end{itemize}

\subsection{PDCP Layer (TS 38.323)}
\begin{itemize}
    \item PDCP Data PDU with 12-bit Sequence Number
    \item PDCP Data PDU with 18-bit Sequence Number
    \item PDCP Control PDU (Status Report)
    \item PDCP Control PDU (ROHC Feedback)
\end{itemize}

\subsection{RLC Layer (TS 38.322)}
\begin{itemize}
    \item RLC Transparent Mode (TM)
    \item RLC Unacknowledged Mode (UM) – 6-bit and 12-bit SN
    \item RLC Acknowledged Mode (AM) – 12-bit and 18-bit SN
    \item RLC AM STATUS PDU
\end{itemize}

\subsection{MAC Layer (TS 38.321)}
\begin{itemize}
    \item MAC Subheader with 8-bit Length field
    \item MAC Subheader with 16-bit Length field
    \item MAC Subheader for fixed-size Control Elements
    \item MAC Subheader with extension bit
\end{itemize}

% -------------------- Quick Start --------------------
\section{Quick Start}

\subsection{Repository Cloning}
\begin{lstlisting}[language=bash]
git clone https://github.com/YOUR_USERNAME/5G-NR-Layer2-PDU-Structures.git
cd 5G-NR-Layer2-PDU-Structures
\end{lstlisting}

\subsection{Compilation and Testing}
\begin{lstlisting}[language=bash]
cd src
gcc test.c -o abc
./abc
\end{lstlisting}

\subsection{Expected Output}
\begin{lstlisting}
=== 5G NR PDU Structure Sizes ===

SDAP Layer:
  SDAP Data PDU (with RQI/RDI):    1 byte(s)
  SDAP Data PDU (without RQI/RDI): 1 byte(s)

PDCP Layer:
  PDCP Data PDU (12-bit SN):       2 byte(s)
  PDCP Data PDU (18-bit SN):       3 byte(s)

=== Total Structures Defined: 24 ===
All 5G NR Layer-2 PDU structures compiled successfully!
\end{lstlisting}

% -------------------- Technical Details --------------------
\section{Technical Details}

\subsection{Bit-Field Ordering}
All bit-fields follow an \textbf{MSB-first} ordering as shown in 3GPP diagrams.  
Note that C bit-field ordering is compiler dependent and must be validated during deployment.

\subsection{Multi-Byte Field Reconstruction}
Sequence numbers spanning multiple bytes are reconstructed as follows:
\begin{lstlisting}
uint16_t sn = (pdu->sn_high << 8) | pdu->sn_low;
\end{lstlisting}

\subsection{Variable-Length PDUs}
The following elements are documented but not fully implemented due to variable length:
\begin{itemize}
    \item PDCP Status Report Bitmap
    \item RLC AM STATUS PDU NACK blocks
    \item MAC Control Element payloads
\end{itemize}

% -------------------- 3GPP References --------------------
\section{3GPP Specification References}

\begin{longtable}{lll}
\toprule
\textbf{Protocol} & \textbf{Specification} & \textbf{Title} \\
\midrule
SDAP & TS 37.324 & Service Data Adaptation Protocol \\
PDCP & TS 38.323 & Packet Data Convergence Protocol \\
RLC & TS 38.322 & Radio Link Control Protocol \\
MAC & TS 38.321 & Medium Access Control Protocol \\
\bottomrule
\end{longtable}

% -------------------- Academic Context --------------------
\section{Academic Context}
This project was developed as part of an advanced wireless communications curriculum.  
It demonstrates:
\begin{itemize}
    \item In-depth understanding of 3GPP specifications
    \item Low-level protocol engineering using C
    \item Systematic modeling of complex protocol stacks
\end{itemize}

% -------------------- Acknowledgments --------------------
\section{Acknowledgments}
\begin{itemize}
    \item 3GPP for comprehensive technical specifications
    \item IIIT Bangalore for academic and research support
    \item The wireless communications research community
\end{itemize}

% -------------------- Project Statistics --------------------
\section{Project Statistics}
\begin{itemize}
    \item Lines of Code: $\sim$1500+
    \item Structures Defined: 24
    \item Protocols Covered: SDAP, PDCP, RLC, MAC
    \item Documentation Coverage: 100\%
\end{itemize}

% -------------------- Citation --------------------
\section{Citation}
\begin{verbatim}
@software{5g_nr_layer2_pdu,
  author = {Sathvika Koppu},
  title  = {5G NR Layer-2 PDU Structure Definitions},
  year   = {2025}
}
\end{verbatim}

\end{document}
